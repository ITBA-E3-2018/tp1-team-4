\newpage

\section*{Ejercicio 3}
\subsection*{Encoder}
Un codificador binario, como su nombre lo indica, codifica la entrada de $2^{n}$ en c\'odigo binario 
de $n$ bits. En un codificador binario sin prioridad, solo una de las entradas debe estar en 1 a la 
vez, y la salida presenta el n\'umero binario que identifica a la entrada que esta en 1. En un codificador 
con prioridad, en cambio, cada entrada tiene asociada una prioridad, y en la salida se mostrara el c\'odigo 
que representa la salida con mayor prioridad en 1. \par
\noindent
Para el dise\'no del encoder, se utilizo la siguiente tabla de verdad:

\begin{center}
    \begin{tabular}{*{5}{C|}|*{3}{C|}}
        \multicolumn{5}{|C}{Inputs}
        \hline
        $x_{3}$ & $x_{2}$ & $x_{1}$ & $x_{0}$ & $enable$ & $z_{1}$ & $z_{0}$ & $flag$ \\
        \hline
        0 & 0 & 0 & 0 & 1 & d & d & 0 \\
        x & x & x & x & 0 & 0 & 0 & 0 \\
        0 & 0 & 0 & 1 & 1 & 0 & 0 & 1 \\
        0 & 0 & 1 & x & 1 & 0 & 1 & 1 \\
        0 & 1 & x & x & 1 & 1 & 0 & 1 \\
        1 & x & x & x & 1 & 1 & 1 & 1 \\
    \end{tabular}    
\end{center}

