\section{Ejercicio 4}

En primer lugar a continuaci�n se presenta la tabla de verdad para la salida, la cual se presenta de la forma Y3, Y2, Y1, Y0 (bit m�s significativo al menos significativo y analogamente para la entrada).
\begin{table}[htb] 
\centering  
\begin{tabular}{|c|c|c|c|}  
\hline  
\multicolumn{4}{|c|}{\textbf{Entradas}} \\ \hline  
\textbf{A3} & \textbf{A2} & \textbf{A1} & \textbf{A0} \\ \hline  
\textbf{0} & \textbf{0} & \textbf{0} & \textbf{0} \\ \hline  
\textbf{0} & \textbf{0} & \textbf{0} & \textbf{1} \\ \hline  
\textbf{0} & \textbf{0} & \textbf{1} & \textbf{0} \\ \hline  
\textbf{0} & \textbf{0} & \textbf{1} & \textbf{1} \\ \hline  
\textbf{0} & \textbf{1} & \textbf{0} & \textbf{0} \\ \hline  
\textbf{0} & \textbf{1} & \textbf{0} & \textbf{1} \\ \hline  
\textbf{0} & \textbf{1} & \textbf{1} & \textbf{0} \\ \hline  
\textbf{0} & \textbf{1} & \textbf{1} & \textbf{1} \\ \hline  
\textbf{1} & \textbf{0} & \textbf{0} & \textbf{0} \\ \hline  
\textbf{1} & \textbf{0} & \textbf{0} & \textbf{1} \\ \hline  
\textbf{1} & \textbf{0} & \textbf{1} & \textbf{0} \\ \hline  
\textbf{1} & \textbf{0} & \textbf{1} & \textbf{1} \\ \hline  
\textbf{1} & \textbf{1} & \textbf{0} & \textbf{0} \\ \hline  
\textbf{1} & \textbf{1} & \textbf{0} & \textbf{1} \\ \hline  
\textbf{1} & \textbf{1} & \textbf{1} & \textbf{0} \\ \hline  
\textbf{1} & \textbf{1} & \textbf{1} & \textbf{1} \\ \hline  
\end{tabular}  
\begin{tabular}{|c|c|c|c|} \hline 
\multicolumn{4}{|c|}{\textbf{Salidas}}                \\ \hline  
\textbf{Y3} & \textbf{Y2} & \textbf{Y1} & \textbf{Y0} \\ \hline  
\textbf{0}  & \textbf{0}  & \textbf{0}  & \textbf{0}  \\ \hline  
\textbf{1}  & \textbf{1}  & \textbf{1}  & \textbf{1}  \\ \hline  
\textbf{1}  & \textbf{1}  & \textbf{1}  & \textbf{0}  \\ \hline  
\textbf{1}  & \textbf{1}  & \textbf{0}  & \textbf{1}  \\ \hline  
\textbf{1}  & \textbf{1}  & \textbf{0}  & \textbf{0}  \\ \hline  
\textbf{1}  & \textbf{0}  & \textbf{1}  & \textbf{1}  \\ \hline  
\textbf{1}  & \textbf{0}  & \textbf{1}  & \textbf{0}  \\ \hline  
\textbf{1}  & \textbf{0}  & \textbf{0}  & \textbf{1}  \\ \hline  
\textbf{1}  & \textbf{0}  & \textbf{0}  & \textbf{0}  \\ \hline  
\textbf{0}  & \textbf{1}  & \textbf{1}  & \textbf{1}  \\ \hline  
\textbf{0}  & \textbf{1}  & \textbf{1}  & \textbf{0}  \\ \hline  
\textbf{0}  & \textbf{1}  & \textbf{0}  & \textbf{1}  \\ \hline  
\textbf{0}  & \textbf{1}  & \textbf{0}  & \textbf{0}  \\ \hline  
\textbf{0}  & \textbf{0}  & \textbf{1}  & \textbf{1}  \\ \hline  
\textbf{0}  & \textbf{0}  & \textbf{1}  & \textbf{0}  \\ \hline  
\textbf{0}  & \textbf{0}  & \textbf{0}  & \textbf{1}  \\ \hline  
\end{tabular} 
\end{table}

Lo siguiente que se realizo fueron los mapas de Karnaugh para cada bit de la salida, para Y0 arriba a la izquierda,para Y1 arriba a la derecha, para Y2 abajo ala izquierda y para Y3 abajo a la derecha:
\noindent \begin{center}
\begin{tabular}{|c|c|c|c|c|}
\hline 
A3A2 & 00 & 01 & 11 & 10\tabularnewline
\hline 
A1A0 &  &  &  & \tabularnewline
\hline 
00 & 0 & 0 & 0 & 0\tabularnewline
\hline 
01 & 1 & 1 & 1 & 1\tabularnewline
\hline 
11 & 1 & 1 & 1 & 1\tabularnewline
\hline 
10 & 0 & 0 & 0 & 0\tabularnewline
\hline 
\end{tabular} %
\begin{tabular}{|c|c|c|c|c|}
\hline 
A3A2 & 00 & 01 & 11 & 10\tabularnewline
\hline 
A1A0 &  &  &  & \tabularnewline
\hline 
00 & 0 & 0 & 0 & 0\tabularnewline
\hline 
01 & 1 & 1 & 1 & 1\tabularnewline
\hline 
11 & 0 & 0 & 0 & 0\tabularnewline
\hline 
10 & 1 & 1 & 1 & 1\tabularnewline
\hline 
\end{tabular} %
\begin{tabular}{|c|c|c|c|c|}
\hline 
A3A2 & 00 & 01 & 11 & 10\tabularnewline
\hline 
A1A0 &  &  &  & \tabularnewline
\hline 
00 & 0 & 1 & 1 & 0\tabularnewline
\hline 
01 & 1 & 0 & 0 & 1\tabularnewline
\hline 
11 & 1 & 0 & 0 & 1\tabularnewline
\hline 
10 & 1 & 0 & 0 & 1\tabularnewline
\hline 
\end{tabular} %
\begin{tabular}{|c|c|c|c|c|}
\hline 
A3A2 & 00 & 01 & 11 & 10\tabularnewline
\hline 
A1A0 &  &  &  & \tabularnewline
\hline 
00 & 0 & 1 & 0 & 1\tabularnewline
\hline 
01 & 1 & 1 & 0 & 0\tabularnewline
\hline 
11 & 1 & 1 & 0 & 0\tabularnewline
\hline 
10 & 1 & 1 & 0 & 0\tabularnewline
\hline 
\end{tabular}
\par\end{center}

Con estos mapas de Karnaugh obtenemos para cada salida las siguientes expresiones en funci�n de los mint�rminos:
$Y0 = m_1 . m_3 . m_5 . m_7 m_{13} . m_{15} . m_9 . m_{11}$ \\
$Y1 = m_1 . m_5 . m_{13} . m_9 + m_2 . m_6 . m_{16} . m_{10}$ \\
$Y2 = m_4 . m_{12} + m_1 . m_3 . m_9 . m_{11} + m_3 . m_2 . m_{11} . m_{10}$\\
$Y3 = m_8 + m_4 . m_5 . m_7 . m_6 + m_1 . m_3 . m_5 . m_7 + m_3 . m_2 . m_7 . m_6$\\
Y si procedemos a simplificar obtenemos:
$Y0 = A0$\\
$Y1 = A0 . \overline{A1} + \overline{A0} . A1$\\
$Y2 = A0 . \overline{A2} + A1 . \overline{A2} + \overline{A0} . \overline{A1} . A2$\\
$Y3 = A0 . \overline{A3} + A1 . \overline{A3} + A2 . \overline{A3} + \overline{A0} . \overline{A1} . \overline{A2} . A3$\\
Y por �ltimo podemos representar las salidas en funci�n de las entradas con las compuertas l�gicas como se puede observar en los gr�ficos que mostramos a continuaci�n, donde no representamos Y0 ya que es directamente igual a la entrada A0:

\includegraphics[{width=\linewidth}]{\string"Y1 Compuertas L�gicas\string".eps}

\includegraphics[{width=\linewidth}]{\string"Y2 Compuertas L�gicas\string".eps}

\includegraphics[{width=\linewidth}]{\string"Y3 Compuertas L�gicas\string".eps}
