%% LyX 2.3.0 created this file.  For more info, see http://www.lyx.org/.
%% Do not edit unless you really know what you are doing.
\documentclass[friulan,spanish,english]{article}
\usepackage[T1]{fontenc}
\usepackage[latin9]{inputenc}
\usepackage{textcomp}

\makeatletter

%%%%%%%%%%%%%%%%%%%%%%%%%%%%%% LyX specific LaTeX commands.
%% Because html converters don't know tabularnewline
\providecommand{\tabularnewline}{\\}

\makeatother

\usepackage{babel}
\addto\shorthandsspanish{\spanishdeactivate{~<>.}}

\begin{document}

\part*{1� PARTE}

Teniendo la expresi�n en maxit�rminos: $f(d,c,b,a)=\prod (M_{0},M_{1},M_{5},M_{7},M_{8},M_{10},M_{14},M_{15})$
Realizaremos una simplificaci�n de esta productoria con �lgebra booleana,
pero antes identificaremos cada t�rmino:

Maxit�rminos: \\
$M_{0}= d+c+b+a\ ;$ 
$M_{1}= d+c+b+\overline{a}\ ;$ 
$M_{5}=d+\overline{c}+b+\overline{a}\ ;$ 
$M_{7}=d+\overline{c}+\overline{b}+\overline{a}\ ;$ \\
$M_{8}=\overline{d}+c+b+a\ ;$
$M_{10}=\overline{d}+c+\overline{b}+a\ ;$
$M_{14}=\overline{d}+\overline{c}+\overline{b}+a\ ;$
$M_{15}=\overline{d}+\overline{c}+\overline{b}+\overline{a}$ 

Ahora pasaremos a realizar la productoria:

$f(d,c,b,a) = M_{0} * M_{1} * M_{5} * M_{7} * M_{8} * M_{10} * M_{14} * M_{15} \\ 
f=(d+c+b+a)*(d+c+b+\overline{a})*(d+\overline{c}+b+\overline{a})*(d+\overline{c}+\overline{b}+\overline{a})* \\ (\overline{d}+c+b+a)*(\overline{d}+c+\overline{b}+a)*(\overline{d}+\overline{c}+\overline{b}+a)*(\overline{d}+\overline{c}+\overline{b}+\overline{a})$

Podemos notar que entre el maxit�rmino $M_{0}$ y $M_{1}$ se puede
aplicar la propiedad de combinaci�n y eliminar la variable $a$ ya
que:

$(d+c+b+a)*(d+c+b+\overline{a})=(d+c+b)$

De la misma forma en los t�rminos $M_{5}$ y $M_{7}$ se puede eliminar
la variable $b$, entre los t�rminos $M_{8}$ y $M_{10}$ se elimina
la variable $b$, y entre los t�rminos $M_{14}$ y $M_{15}$ se elimina
la variable $a$. Entonces se puede eliminar algunas variables y reducir
la ecuaci�n a:

$f(d,c,b,a) = (d+c+b)*(d+\overline{c}+\overline{a})*(\overline{d}+c+a)*(\overline{d}+\overline{c}+\overline{b})$

Para no complicar tanto la ecuaci�n y no enredarnos vamos a separar
la funci�n principal en 2, para que $f(d,c,b,a)=f_{1}(d,c,b,a)*f_{2}(d,c,b,a)$,
definiendo a cada una de la siguiente forma:

$f_{1}(d,c,b,a)=(d+c+b)*(\overline{d}+\overline{c}+\overline{b})$
y $f_{2}(d,c,b,a)=(\overline{d}+c+a)*(d+\overline{c}+\overline{a})$

A partir de ahora analizaremos solo $f_{1}$ y veremos el resultado
que se obtiene. No haremos el an�lisis para $f_{2}$ ya que ser� el
mismo pero con un resultado final distinto. 

$f_{1}=(d+c+b)*(\overline{d}+\overline{c}+\overline{b})=d*\overline{c}+d*\overline{b}+c*\overline{d}+c*\overline{b}+b*\overline{d}+b*\overline{c}+d*\overline{d}+c*\overline{c}+b*\overline{b}$

Vamos a recordar un teorema de 1 variblae $X*\overline{X}=0$ y a
reordenar los t�rminos:

$f_{1}= b*\overline{c} + \overline{c}*d + \overline{b}*d + b*\overline{d} +  \overline{d}*c + \overline{b}*c $

Para poder seguir reduciendo la ecuaci�n vamos a utilizar la propiedad
del consenso $(x+y)*(y+z)*(\overline{x}+z)=(x+y)*(\overline{x}+z)$
con los primeros 3 t�rminos, y luego con los segundos 3 t�rminos,
entonces:

$b*\overline{c}+\overline{c}*d+\overline{b}*d = b*\overline{c}+\overline{b}*d$ \ ;\  $b*\overline{d}+\overline{d}*c+\overline{b}*c = b*\overline{d} + \overline{b}*c$

$f_{1}=b*\overline{c} + \overline{b}*d + b*\overline{d} + \overline{b}*c=(\ b*(\overline{c}+\overline{d})\ +\ \overline{b}*(c+d)\ )$

De la misma forma, para $f_{2}$ quedar�:

$f_{2}=a*\overline{c} + a*\overline{d} + \overline{a}*c + \overline{a}*d =(\ a*(\overline{c}+\overline{d})\ +\ \overline{a}*(c+d)\ ) $

Volviendo a $f$ obtenemos:

$f=f_{1}*f_{2}=(\ b*(\overline{c}+\overline{d})\ +\ \overline{b}*(c+d)\ )*(\ a*(\overline{c}+\overline{d})\ +\ \overline{a}*(c+d)\ )$

Aplicando la propiedad distributiva:

$f=ab(\overline{c}+d)(\overline{c}+\overline{d})+a\overline{b}(c+d)(\overline{c}+d)+\overline{a}b(c+\overline{d})(\overline{c}+\overline{d})+\overline{a}\overline{b}(c+\overline{d})(c+d)$

Para reducir estos t�rminos utilizamos la propiedad de combinaci�n
de 2 variables, ya que $(X+Y)(X+\overline{Y})=X$:

$(\overline{c}+d)(\overline{c}+\overline{d})=\overline{c} \ ;\ (c+d)(\overline{c}+d)=d\ ;\ (c+\overline{d})(\overline{c}+\overline{d})=\overline{d}\ ;\ (c+\overline{d})(c+d)=c$

Entonces como resultado final obtenemos que:

$f(d,c,b,a)=ab\overline{c}+a\overline{b}d+\overline{a}b\overline{d}+\overline{a}\overline{b}c=\overline{d}b\overline{a}+d\overline{b}a+\overline{c}ba+c\overline{b}\overline{a}$

\part*{2� PARTE}

Ahora veremos que suceder�a si vieramos el problema desde los mapas
de Karnaugh, teniendo los mismos maxit�rminos.

\begin{tabular}{|c|c|c|c|c|}
\hline 
dc \textbackslash{} ba & 00 & 01 & 11 & 10\tabularnewline
\hline 
00 & 0 & 1 & 1 & 0\tabularnewline
\hline 
01 & 0 & 0 & 1 & 1\tabularnewline
\hline 
11 & 1 & 0 & 0 & 1\tabularnewline
\hline 
10 & 1 & 1 & 0 & 0\tabularnewline
\hline 
\end{tabular}

Agrupamos todos los maxit�rminos de a pares verticales, $M_{0}$ con
$M_{1}$, $M_{5}$ con $M_{7}$, $M_{8}$ con $M_{10}$ y $M_{14}$
con $M_{15}$. Al primer par lo llamaremos $I_{1}$, al segundo $I_{2}$,
al tercero $I_{3}$ y el �ltimo lo llamaremos $I_{4}$

\part*{3 PARTE }

\selectlanguage{friulan}%
DaDo el resultado final, el circuito utilizando solo compuertas NOT,
AND y OR es el siguiente:

\begin{figure}
\caption{\foreignlanguage{english}{TP1EJ2c\_electroiii}}

\selectlanguage{english}%
\selectlanguage{friulan}%
\end{figure}

\selectlanguage{english}%

\part*{4 PARTE }

\selectlanguage{spanish}%
Para utilizar solo compuertas NAND por ser el grupo 4, necesitamos
trabajar sobre $f(d,c,b,a)$ viendo que si se niega 2 veces la ecuaci�n
final para mantener la igualdad, luego de aplicar el teorema de De
Morgan, obtendremos un resultado que puede tratarse de un conjunto
de compuertas NAND y NOT:

$\overline{\overline{f}}=\overline{\overline{\overline{d}b\overline{a}+d\overline{b}a+\overline{c}ba+c\overline{b}\overline{a}}}=\overline{\overline{(\overline{d}b\overline{a})} * \overline{(d\overline{b}a)} * \overline{(\overline{c}ba)} * \overline{(c\overline{b}\overline{a})}}$

Analizando el nuevo resultado final podemos notar que se trata solo
de productos negados y algunas entradas negadas, siendo as� necesarias
5 compuertas NAND's y 4 NOT's, o simplemente 9 NAND's (el ejercicio
solo pide NAND). El circuito ser�a el siguiente:

\begin{figure}
\selectlanguage{english}%

\selectlanguage{spanish}%
\caption{TP1EJ2d\_electroiii}

\selectlanguage{english}%
\selectlanguage{spanish}%
\end{figure}
\selectlanguage{english}%

\end{document}
